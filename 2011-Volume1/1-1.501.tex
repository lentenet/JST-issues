%!TEX TS-program = xelatex
%!TEX encoding = UTF-8 Unicode

\input{jst-style}
\def\bookby{Christian Smith}
\def\booktitle{The Bible Made Impossible: Why Biblicism is Not a Truly Evangelical Reading of Scripture}
\def\shorttitle{The Bible Made Impossible}
\def\bookpublication{Grand Rapids, Michigan: Brazos Press, 2011. Pp. xiv + 220. Hardcover. \$22.99.}
\def\keywords{evangelicalism, Catholicism, hermeneutics}
\def\reviewauthor{Wesley A. Hill}
\def\institution{Durham University, UK}
\def\datesubmitted{30 July 2011}
\def\dateaccepted{25 October 2011}
\def\datepublished{26 October 2011}
\def\jstyear{2011}
\def\jstvol{1}
\def\jstiss{1}
\def\firstpage{501}
\def\lastpage{504}
\input{begin-jst-review}
\noindent Christian Smith, professor of sociology and director of the Center for the Study of Religion and Society at the University of Notre Dame, best known perhaps for the award-winning book he co-authored with Michael Emerson, \marginpar{Michael O. Emerson and Christian Smith, \emph{Divided by Faith: Evangelical Religion and the Problem of Race in America} (Oxford University Press, 2001).} has turned his attention to evangelical use of Scripture in his most recent study. The title discloses the two halves of the book’s argument. Evangelical “biblicism” (to be defined below) has made the Bible “impossible,” has rendered it unable to achieve the work it was designed to do. But an “evangelical” reading of the Bible, one that is oriented at all times to the \emph{evangel}, the Christ-event, avoids the problems of biblicism and allows Scripture to function rightly in the life of the church.

Smith, a recent convert to Catholicism, writes as a former evangelical concerned that evangelicals be able to accept his argument on their own terms, without having recourse to Catholic understandings of Scripture vis-à-vis the Magisterium. Accordingly, his project targets evangelical examples of biblicism, by which he refers to “a particular theory about and style of using the Bible that is defined by a constellation of related assumptions and beliefs about the Bible’s nature, purpose, and function” (p. 4). That theory and style is comprised of ten interlocking features: (1) the Bible is divine writing such that its words are God’s words; (2) it is a total representation of God’s communication, (3) including complete coverage of the divine will; (4) it is perspicuous and thus open to “democratic” interpretation, (5) which proceeds by way of commonsense hermeneutics, (6) leading to a doctrine of “solo” \emph{scriptura} (Scripture “denuded,” unclothed by creeds and confessions); and (7) because the Bible possesses internal harmony (8) and universal applicability, (9) therefore, an inductive method is best suited to unearthing the Bible’s meaning, (10) which suggests a handbook model whereby the Bible is mined for its “positions” on everything from dating to gardening.

By adopting this biblicist account of the Bible and its function in Christian thought and life, evangelicals must necessarily also be committed to the notion that the Bible gives clear and certain purchase on divine revelation. But, Smith notes, the opposite proves to be true in practice. While appealing to the same Bible, different evangelical readers arrive at wildly divergent construals not only of peripheral matters—such as gender roles in the church or the right mode of baptism—but also of major, central teachings on christology and soteriology, to name only two contested arenas. “Knowledge of ‘biblical’ teachings, in short, is characterized by \emph{pervasive interpretive pluralism}” (p. 17, emphasis original).

For Smith, pervasive interpretive pluralism should force evangelicals to reconsider what they think about the nature of the Bible itself. “[O]n important matters the Bible apparently is not clear, consistent, and univocal enough to enable the best-intentioned, most highly skilled, believing readers to come to agreement as to what it teaches” (p. 25). Therefore, rather than being the product of interpreters’ misunderstandings, willful or otherwise, of a clear revelation, the interpretive pluralism Smith documents exists “because the texts themselves are multivocal, polysemic, and multivalent in character” (p. 50; cf. p. 142). Seeing the diversity of opinions on almost every major issue of doctrine and practice should lead evangelicals to recognize the messy, pluriform character of the Bible.

What is needed, then, is a better theological account of what kind of text Scripture is and therefore what we can expect from it. Here Smith enlists Barth’s help (pp. 121--6) to argue that the Bible is a collection of differing voices that may be heard in concert insofar as they are heard to be witnesses of God’s singular and saving act in Jesus Christ. “This means that we always read Scripture Christocentrically, christologically, and christotelically, as those who \emph{really} believe what the Nicene and Chalcedonian creeds say” (p. 98, emphasis original). Adopting this evangelical (in the root sense of \emph{evangel}) rule would enable evangelicals to eschew the handbook approach, wherein Scripture is mined for advice on various and sundry concerns, and to concentrate instead on understanding the Bible as testimony to the saving significance of Jesus. Addressing the concerns evangelicals (rightly) wish to address must involve examining those concerns in light of God’s saving deed in Christ rather than in light of individual texts collected from the pages of the Bible and treated as timeless truths (p. 111).

Readers may well wonder what is new in this proposal—or for that matter in the critique that comprises the book’s first half. Part of the originality of Smith’s argument may be that popular evangelical instances of biblicism (e.g., the T-shirt slogan “BIBLE—Basic Instruction Before Leaving Earth” or books with titles like \emph{The World According to God: A Biblical View of Culture, Work, Science, Sex, and Everything Else}; see pp. 6--12) are traced back to institutional, confessional, and scholarly contexts that allegedly give rise to, or at least do nothing to check, them. So, immediately after noting the problems of appealing to Bible texts to yield, say, a coherent Christian account of business ethics, Smith critiques the Westminster Confession of Faith and the 1978 Chicago Statement on Biblical Inerrancy, among other statements of belief adopted by institutions like Wheaton College and Trinity Evangelical Divinity School, as examples of biblicism (p. 14). His point seems to be that these confessions and statements are organically related to the more popular forms of biblicism he cites.

Smith does acknowledge, on the basis of a survey of required textbooks at a range of evangelical seminaries and colleges, that biblicism as he defines it is \emph{not} “taught directly by most faculty and evangelical seminaries and divinity schools” (p. 185 n. 36). He also qualifies his argument by acknowledging his definition of biblicism as a synthetic, summary one, and not all targets of critique would hold to each of its ten points. But despite the caveats, readers may still find themselves unsure about how far Smith’s critique is meant to extend. He does not engage extensively with those who have labored to affirm some aspects of what he terms “biblicism” in confessionally and historically rooted ways. The work of Timothy Ward on the sufficiency of Scripture (see his \emph{Word and Supplement: Speech Acts, Biblical Texts, and the Sufficiency of Scripture}), for instance, doesn’t appear in Smith’s bibliography, nor is Kevin Vanhoozer’s recent rehabilitation of an evangelical “Scripture principle” (in, e.g., \emph{The Drama of Doctrine}) a prominent conversation partner in the book. One is left wondering whether Smith deems these proposals for an identity between the (variegated) words of Scripture and God’s own speech to have failed—and how his argument might have been complicated or enriched had he chosen to engage them.

At a deeper level, however, it is simply not true that the pervasive interpretive pluralism Smith takes to be the deadly fruit of a biblicist model is only a problem for evangelicals and therefore directly related to evangelical biblicism. The doctrinal differences among evangelicals that Smith highlights (church polity, free will and predestination, the “fourth commandment,” atonement and justification, charismatic gifts, etc.; see pp. 28--36) are debated in Christian churches that would not identify as Protestant, let alone evangelical. Such disagreement seems endemic to \emph{Christian} thinking and living, and not just \emph{biblicist} thinking. By the same token, Smith’s own church, the Roman Catholic Church, affirms both the divine inspiration and inerrancy of Scripture (see the \emph{Catechism of the Catholic Church}, Article 3.2.108) and Scripture’s Christocentric character (Article 3.2.108)—and also adopts highly specific positions on women’s ordination, clerical celibacy, the morality of contraceptives, to name only a few issues about which many evangelicals would disagree. Certainly the Catholic church wishes to affirm these positions by a train of theological reasoning that precludes appeal to Scripture alone. However, the point stands that taking on board a Christocentric hermeneutic does not by itself pressure the church to abandon a position on issues about which there is, arguably, diversity, multivocality, and polysemy in the biblical texts. In that light, can Smith’s constructive proposal achieve all that he wishes for it to achieve? Apparently not, at least not without entering into precisely the question that thoughtful evangelicals regularly pose: How do we know what a “christological,” “evangelical” (i.e., gospel-centric) approach to, say, women’s ordination looks like without engaging in the inherently problematic, complicated task of hearing Scripture’s \emph{own} voice on that issue from the perspective of the Christ-event? In other words, it does not seem that asking about Scripture’s viewpoint(s) on a given issue \emph{x} really is so neatly separable from asking what it means to think about that issue christologically, as Smith’s argument implies.

In short, Smith’s book raises questions any serious Christian reader of the Bible will have to face. Yet due to its vague starting point (how extensive, exactly, is the tribe of the “biblicists”?) and its equally vague constructive proposal (how does reading the Bible “christologically” solve the problem of pervasive interpretive pluralism, if virtually every thoughtful Christian, including many of the “biblicists” he critiques, already agrees with that aim?), the book is unlikely to achieve what is laudable in its goals or to reach those who most need to give up some of the practices Smith decries.
\end{document}
